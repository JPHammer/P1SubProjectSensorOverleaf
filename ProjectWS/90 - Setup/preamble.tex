\usepackage[utf8]{inputenc} % symboler, såsom æøå eller lignende
\usepackage{textcomp} % flere symboler, såsom €
\usepackage{fontawesome}
\usepackage{graphicx} % Muliggør brug af figurer
\usepackage{float} % Muliggør positionering af figurer
\usepackage{wrapfig} % Muliggør tekst og figur på samme linje
\usepackage{caption}
\usepackage{subcaption} % Muliggør subfigures
\usepackage{hyperref} % interaktive referencer!
\usepackage{csquotes} % bruges af babel pakken
\usepackage[english]{babel} % rapportens sprog. Skal også rettes i "main.tex" under "\selectlanguage"
\usepackage[font = small, labelfont = bf]{caption} % til mindre figur tekster hvor figur nummeret er fremhævet med fed
\usepackage{xcolor} % Til brugerdefinerede farvekoder
\usepackage{mathptmx} % Til brugerdefinerede font størrelser
\usepackage{pdfpages} % for an kunne importere hele pdf sider
\usepackage{booktabs} % for at definere /toprule og /midrule
\usepackage{hyperref} %Muliggør brug af hyperref

% load a colour package
\usepackage{xcolor}
\definecolor{aaublue}{RGB}{33,26,82}% dark blue
\usepackage{graphicx} % Set up how figure and table captions are displayed

\usepackage[font = small, labelfont = bf]{caption} % til mindre figur tekster hvor figur nummeret er fremhævet med fed
\usepackage{xcolor} % Til brugerdefinerede farvekoder
\usepackage{mathptmx} % Til brugerdefinerede font størrelser
\usepackage{pdfpages} % for an kunne importere hele pdf sider
\usepackage{booktabs} % for at definere /toprule og /midrule
\usepackage{ulem} %math
\usepackage{amsmath} %Math
\usepackage{xcolor,colortbl} % farver i tabeller
\usepackage{multicol}%kolonner
\hyphenpenalty=100000 % *EKSPERIMENTIEL* Forhindrer automatisk deling af ord

% ------ Marginer ------
\usepackage[left = 2cm, right = 2cm, bottom = 3cm ,top = 3.5cm]{geometry} % juster selv marginerne ved at indtaste andre tal

% ------ Farver ------
\definecolor{chapterNumColor}{RGB}{150, 150, 150} % 255, 255, 255 er helt hvid. 0, 0, 0 er helt sort

% ------ Lister ------
% ducomentation for list options: http://www.texnia.com/archive/enumitem.pdf
\usepackage{enumitem} % til redigering af lister
\setlist{itemsep = 0.5cm, itemindent = 0cm, labelsep = 0.5cm, leftmargin = 0.5cm} % globale indstillinger for lister.

% ------ Referencer ------
% Biblatex cheat sheet: http://tug.ctan.org/info/biblatex-cheatsheet/biblatex-cheatsheet.pdf
\usepackage[style=numeric, sorting=none]{biblatex}
\appto{\bibsetup}{\raggedright}
\addbibresource{ProjectWS/referencer.bib} % fortæller hvad vores reference bibliotek hedder og hvor det er at finde i mappestrukturen
\DeclareFieldFormat{postnote}{s. #1} % ændrer "page" til "s." i kilder ved enkelte sider
\DeclareFieldFormat{multipostnote}{s. #1} % ændrer "pages" til "s." i kilder ved interval af sider

% Sørger for at der står et al. når der er flere end to forfattere.
\DefineBibliographyStrings{danish}{
  andothers = {et\addabbrvspace al\adddot}
}

% ------ Indholdsfortegnelse (Table of contents (TOC)) ------
\usepackage{tocloft}
\setcounter{secnumdepth}{3} % jo højere tallet er, jo "dybere" overskrifter bliver nummereret. 3 giver Chapter, section, subsection og subsubsection
\setcounter{tocdepth}{1} % dybden af indholdfortegnelsen. 1 viser chapter og section
% \renewcommand{\cftpartleader}{\cftdotfill{\cftdotsep}} % prikker for parts
% \renewcommand{\cftchapleader}{\cftdotfill{\cftdotsep}} % prikker for chapters
% \renewcommand{\cftsecleader}{\hfill} % fjern prikker for sections
% \renewcommand{\cftsubsecleader}{\hfill} % prikker for subsections

% ------ Sidehoved og sidefod ------
\usepackage{lastpage} % Gør det muligt at få vist sidenummer for sidste side i rapporten.
\usepackage{fancyhdr} % Pakke til at modificere sidehoved og sidefod
\fancyhf{} % Sætter header og footer til ikke at indeholde noget - så vi er klar til at give det nyt indhold
\setlength{\headheight}{15pt} % Bestemmer hvor langt nede sidehovedet vises
\fancyfoot[R]{\thepage \ af \pageref{LastPage}} % Viser sidenummeret nederst til højre. Ændr "R" for at få det placeret et nyt sted
\fancyhead[L]{\leftmark} % placerer sidehoved indhold i øverste venstre side

\fancypagestyle{plain} % Brugerdefineret side når styletypen plain er valgt. I dette tilfælde er det hver gang et kapitel opstår. 
{
    \fancyhf{} % Sætter header og footer til ikke at indeholde noget - så vi er klar til at give det nyt indhold
    \renewcommand{\headrulewidth}{0pt} % Fjerner linjen øverst under sidehovedet
    \fancyfoot[R]{\thepage \ af \pageref{LastPage}} % Viser sidenummeret nederst til højre. Ændr "R" for at få det placeret et nyt sted
}
\pagestyle{fancy}

% ------ Style for kapitlers sidehoved ------
\usepackage{titlesec} % Til brugerdefinerede sidehoved
\titleformat{\chapter}{\fontsize{35pt}{35pt}\bfseries\color{black}}{{\color{chapterNumColor}\thechapter}}{15pt}{} % sætter fonttypen, skriftstørrelsen og farven på kapitel
\titleformat{\part}{\fontsize{40pt}{45pt}\bfseries\centering\color{black}}{{\color{chapterNumColor}\thepart}}{15pt}{}[\thispagestyle{empty}\addtocounter{page}{-1}] % sætter fonttypen, skriftstørrelsen og farven på parts

% ------ Selvlavede kommandoer til citater ------

\newcommand{\centerQuote}[3] % Indholdet i {} er navnet på den kommando der er lavet
{
    \begin{quote} % Sørger for at centrere dit citat
        \textit{"#1"} - \parencite[#2]{#3} % Laver formatet "Citat her" - (Efternavn, Årstal, s. sidetal)
    \end{quote} 
}

\newcommand{\inlineQuote}[3] % Indholdet i {} er navnet på den kommando der er lavet
{\textit{"#1"} \parencite[#2]{#3}} % Laver formatet "Citat her" - (Efternavn, Årstal, s. sidetal)

\usepackage{lipsum} % til at generere dummy text. Brug \lipsum[2] for at vise paragraf nr 2 i Lorem Ipsum. Brug \lipsum[4-7] for at vise paragraf 4 til 7 i Lorem Ipsum.

% ------ ToDo Notes ------
%ToDo Notes cheat sheet: https://mirrors.dotsrc.org/ctan/macros/latex/contrib/todonotes/todonotes.pdf
\setlength{\marginparwidth}{2cm} % Retter fejl med marginer i ToDo notes.
\usepackage{todonotes} % Muliggør brug af "ToDo notes" funktionalliteter

%------ Glossary ------
%Glossaries cheat sheet: https://mirrors.dotsrc.org/ctan/macros/latex/contrib/glossaries/glossariesbegin.pdf
\usepackage{glossaries} % Muliggør brug af "glossaries" funktionen
\makenoidxglossaries % Muliggør udskift af glossary-list i dokumentet
\glstoctrue % Inkluderer glossary i toc
\setacronymstyle{long-short}

\newacronym{svm}{SVM}{support vector machine} % Muliggør brug af glossary.tex for samlede definitioner